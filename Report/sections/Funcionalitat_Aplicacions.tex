\section{Introducció i Funcionalitat}

El radio altímetre és un dispositiu electrònic essencial posicioant a bord d'una aeronau destinat a \textbf{proporcionar mesures acurades de l'altitud absoluta a la que es troba l'aeronau respecte el terreny}. 
Aquests, estàn dissenyats per a funcionar durant tota la vida útil de l'aeronau en la cual están instal·lats (uns 30 anys) i per tant per fer front un ampli rang d'operacions tenint en compte les possibles tolerancies produides per l'edat dels equips.

Els sistema de radio altímetre d'una aeronau esta format per tres tranceptors idèntics i l'equipament associat a cada un d'ells. Totes tres unitats operen de manera simultania i independent en la banda de 4.2 - 4.4 GHz (banda aeronautica destinada exclusivament per funcions de radionavegació) emetent un senyal i processant el senyal rebotat rebut en forma d'alçada.

Actualment, existeixen dos tipus d'altímetres que difereixen en el sistema de modulació del senyal emès: 
\begin{itemize}
\item Ona continua modelada en freqüència \textit{(FMCW)}
\item Modulació de pols \textit{(pulse modulation)}
\end{itemize}







